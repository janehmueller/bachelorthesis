% (1 Seite)
\section{Related Work}
\label{sec:RelatedWork}
Both NER and EL are widely known and researched problems. Recognizing businesses in texts was one of the first research topics for NER according to Nadeau et al.\ \cite{ner-sekine2007}. This approach by Rau uses rules and heuristics to recognize businesses in texts\ \cite{rau91}. Mikheev et al. use a similar approach, also using rules and heuristics and not needing gazetteers, and extend the set of recognized entities by persons and locations\ \cite{Mikheev:1999:NER:977035.977037}. Ritter et al. recognize named entities in Tweets by using part-of-speech tagging and chunking\ \cite{Ritter:2011:NER:2145432.2145595}. Approaches to NER in the German language are, e.g., done by Blessing et al., who recognize and ground German geographic proper names by using a three-step model consisting of spotting, typing and referencing\ \cite{Blessing:2007:TCM:1316948.1316956}. There are also approaches for NER on businesses in German texts. One such approach by Loster et al. uses various dictionaries, regular expressions and text contexts to recognize businesses\ \cite{Michael2017a}.\par
There are also many research efforts solving the EL problem in various ways. Some combine the EL with the NER while others assume that the entities were already recognized and tagged. Sil et al. present an approach that combines EL with NER by combining candidate mentions from NER systems and candidate entity links from EL systems and making joint predictions\ \cite{Sil:2013:RJN:2505515.2505601}. Brauer et al. also combine NER and EL\ \cite{Brauer:2008:MEE:1458550.1458566}. They recognize and associate entities in unstructured data with those in structured data by disambiguating mappings of entities in the text to entities in the structured data by exploiting relationships from the structured data and the documents' structure. Grütze et al. propose a system that does only the EL task and assumes the NER to be already done\ \cite{coheel}. Their focus lies on the reliability and performance of the Entity Linking, similar to the approach this thesis proposes. They annotate millions of documents in an acceptable amount of time and focus on the precision of their EL as well. Additionally, they improve the recall by using a second classifier with a high recall and combining the candidates of both with Random Walks.
