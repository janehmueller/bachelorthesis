\begin{abstract}{Abstract}
To create a graph of businesses and their relations with each other one can use structured knowledge bases, like Wikidata and DBpedia. But these knowledge bases contain incomplete information about the relations of businesses. To find relations not contained in these knowledge bases they must be extracted from unstructured texts like newspaper articles. To do this, we must first recognize businesses in unstructured texts and link them to entities in our knowledge base.\par
This thesis proposes a concept that combines both Named Entity Recognition and Entity Linking of businesses in German texts. It uses only publicly available data, i.e., the German Wikipedia and Wikidata, to train a classification model that annotates and links unstructured documents to entities in our knowledge base.\par
Our approach annotates documents in an efficient way and with a high reliability. With it, we are able to annotate millions of documents in an acceptable amount of time and can, therefore, annotate newly published newspaper articles in real-time. This enables us to update our knowledge base and the business graph with up-to-date relations.\par
\end{abstract}
% \newpage
\begin{abstract}{Zusammenfassung}
Um einen Graphen von Unternehmen und ihren Beziehungen untereinander zu erstellen, kann man strukturierte Wissensbasen, wie z.B. Wikidata oder DBpedia, benutzen. Diese enthalten aber keine vollständigen Informationen über die Beziehungen von Unternehmen. Um die nicht enthaltenen Beziehungen zu finden, muss man sie aus unstrukturierten Texten wie Zeitungsartikeln extrahieren. Um dies zu tun müssen wir erst Vorkommen von Unternehmen in unstrukturierten Texten finden und diese zu Entitäten in unserer Wissensbasis verlinken.\par
Die vorliegende Bachelorarbeit stellt ein Konzept vor, dass sowohl die Named Entity Recognition als auch das Entity Linking von Unternehmen in deutschen Texten kombiniert. Es benutzt nur öffentlich zugängliche Daten, Wikipedia und Wikidata, um ein Klassifikationsmodell zu trainieren, dass unstrukturierte Dokumente annotiert und zu Unternehmen in unserer Wissensbasis verlinkt.\par
Unser Ansatz annotiert die Dokumente effizient und mit einer hohen Zuverlässigkeit. Damit ist es möglich Millionen von Dokumenten in einem akzeptablen Zeitraum zu annotieren. Dadurch kann man neu veröffentlichte Zeitungen in Echtzeit annotieren und die Beziehungen zwischen Unternehmen extrahieren. Dies ermöglicht es uns, unsere Wissensbasis und unseren Unternehmensgraphen aktuell zu halten.\par
\end{abstract}