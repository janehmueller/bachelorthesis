\section{Ambiguous Business Aliases}
\label{sec:IntroABA}
The graph of Germany's corporate landscape contains nodes, which are businesses, and edges, which are the relations between businesses. Information about businesses is extracted from structured knowledge bases such as Wikidata and DBpedia. They describe the businesses themselves but contain incomplete information about the relations between them. These relations are described in unstructured texts like Wikipedia or newspaper articles. They need to be extracted from these unstructured texts to be transformed into an edge of the graph.\par
For our approach to extract a relation between two businesses they must first be mentioned in the same sentence. Then both of these mentions need to be found and linked to the entities representing these businesses. Finding these mentions is called \textit{Named Entity Recognition} (NER). They must then be linked to the correct entities in the knowledge base. This step is called \textit{Entity Linking} (EL). The last step is to extract the relation between the two businesses from the sentence, which is called \textit{Relation Extraction} (RE). Our approach combines NER and EL into a single step and transforms it into a classification problem. These resulting two steps, the combination of NER and EL and the RE step, make up the Information Extraction step in the pipeline described in the Introduction. Janetzki \cite{janetzki} describes the creation of our knowledge base and the extraction of features used to classify mentions. This work evaluates the quality of the features and different classification models. Schneider \cite{schneider} describes and evaluates different Relation Extraction methods.\par
Section \ref{sec:RelatedWork} covers related work in the field of both NER and EL and especially related work about recognizing and linking business entities and efficient NER and EL on millions of documents in acceptable time. Section \ref{sec:NEL} describes the used approach that combines both NER and EL. After that different classification models and configurations are compared and evaluated in Section \ref{sec:ModelEval}. Section \ref{sec:FeatureEval} evaluates the features used to train the classifier. These features are described in detail by Janetzki \cite{janetzki} and Grütze et al. \cite{coheel}. Section \ref{sec:NELEval} discusses the performance and quality of this approach when annotating millions of newspaper articles. Finally, Section \ref{sec:Conclusion} concludes this thesis and describes possible improvements that could be done in the future.
